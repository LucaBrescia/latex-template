\documentclass[titlepage]{report}
\usepackage[italian]{babel}
\usepackage[babel]{csquotes}
\usepackage[a4paper,left=2cm,bottom=2cm,right=2cm,top=2.5cm]{geometry}
\usepackage{subcaption}
\usepackage{url}
\usepackage{graphicx} 			
\usepackage{sidecap}		
\usepackage{setspace}		
\usepackage{textcomp} 			
\usepackage[table]{xcolor}		
\usepackage{float}
\usepackage{hyperref}
\usepackage{fancyhdr}
\usepackage[nowrite,infront,standard,swapnames]{frontespizio}
\usepackage{amsmath}            % subequations
\usepackage{listings}           % code listing style    
\usepackage[most]{tcolorbox}
\usepackage{xcolor}
\usepackage[nowrite,infront,standard,swapnames]{frontespizio}
%% Definizione nomi colori
\definecolor{codegreen}{rgb}{0,0.6,0}
\definecolor{codegray}{rgb}{0.5,0.5,0.5}
\definecolor{codepurple}{rgb}{0.58,0,0.82}
\definecolor{backcolour}{rgb}{0.98,0.98,0.95}

%% Stile per il linsting
\lstdefinestyle{customCode}{
    backgroundcolor=\color{backcolour},   
    commentstyle=\color{codegreen},
    keywordstyle=\color{magenta},
    otherkeywords={QBluetoothLocalDevice,QString,QOverload,QDialog,QWidget,Ui_Chat,QPushButton},
    numberstyle=\tiny\color{codegray},
    stringstyle=\color{codepurple},
    basicstyle=\ttfamily\footnotesize,
    breakatwhitespace=false,         
    breaklines=true,                 
    captionpos=b,                    
    keepspaces=true,                 
    numbers=left,                    
    numbersep=5pt,                  
    showspaces=false,                
    showstringspaces=false,
    showtabs=false,                  
    tabsize=2
}


    

\lstdefinestyle{shelllike}{
    language=bash,
    backgroundcolor=\color{black},
    basicstyle=\ttfamily\color{white}, commentstyle=\color{green}\ttfamily,
    keywordstyle=\color{blue}\ttfamily, showstringspaces=false,
    aboveskip=0pt,
    belowskip=0pt
}

% Set link border color none. See more options: https://en.wikibooks.org/wiki/LaTeX/Hyperlinks
\hypersetup{hidelinks}

%% Applico lo stile ai listing
\lstset{style=customCode}

\begin{document}
    % Per saltare una riga alla fine del paragrafo
    \setlength{\parskip}{\baselineskip} 
    % Per creare il listing personalizzato "bashcode"
    \newtcblisting{bashcode}{
        arc=0pt,
        boxrule=0pt,
        colback=black,
        listing only,
        listing options={
            style=shelllike,
            numbers=none
        },
        right=0pt,
        top=0pt,
        bottom=0pt,
        boxsep=5pt,
        enhanced,
        overlay={
            \begin{tcbclipframe}
                \fill[black, rounded corners] (frame.south west) rectangle ([xshift=-3mm]frame.north west);
            \end{tcbclipframe}
        },
    }

    % Compile the *-frn.tex file with the command "pdflatex" before this file
    \begin{frontespizio}
        \Universita {University Name}
        % Uncomment next line to add a image file to your Frontespizio
        %\Logo [Size]{Path}
        \Dipartimento {Department}
        \Corso [Course Level]{Course Name}
        \Annoaccademico {Academic Year}
        \Titoletto{Uptitle}
        \Titolo {Title}
        \Sottotitolo {Subtitle}
        \NCandidato{Label}
        \Candidato [Serial]{Name}
    \end{frontespizio}

    \tableofcontents 

    \chapter*{Abstract}
    \label{ch: Abstract}
    \addcontentsline{toc}{chapter}{Abstract}

    \chapter{Chapter 1}
    \label{ch: Chapter 1}

\end{document}